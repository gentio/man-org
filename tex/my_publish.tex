\documentclass{article}
\usepackage{xeCJK}
\usepackage{graphicx}
\usepackage{caption}
\usepackage[top=1.5,bottom=2cm,left=1.68cm,right=1.68cm]{geometry}
\begin{document}
\title{My tex templete}
\author{Gentio}
\maketitle

\section{the templete}
\begin{verbatim}
%\documentclass{article}
%\usepackage{xeCJK}
%\usepackage{graphicx}
%\usepackage{capation}
%\usepackage{top=1.5,bottom=2cm,left=1.68cm,right=1.68cm}{geometry}

%\begin{document}
%\title{A title}
%\author{A author}
%\maketitle

%\section{Section One}
%something

%\section{Section Two}

%\begin{center} % make the text centering in the tex.
%\end{center}
%\textbf{some words} % make the text more strongger.
%\vspace{1cm} % make text have a space line high 1cm
%\includegraphics[scale=0.5]{your_pitcure.png|jpg} % insert the image and you can setting the other argument like width and height.
%\include{something.tex}
%\input{other.tex} % the difference of them is that include make a new page after give a new page.
%\begin{enumerate} % give a item list
%\item one item
%\item two item
%\end{enumerate}

%\subsection{Subsection one}
% \begin{verbatim} % to give a code envirenment in the tex.
% \end{verbatim}
\end{verbatim}

\section{some math funtion to write:}
$ a^2 = b^2 + c^2 $

$ \frac{2}{\sqart{2}} $
make the math funtion like
\begin{verbatim}
%\begin{tabular}{p{1cm}|c|l|r}
\hline
0 & 1 & 2 & 3 & 4  \\ \hline
6 & 7 & 8 & 9 & 10  \\ \hline
%\end{tabular}
\end{verbatim}
%
%\begin{tabular}{c|l|r}
%\hline
%0 & 1 & 2 & 3 & 4  \\ \hline
%6 & 7 & 8 & 9 & 10  \\ \hline
%\end{tabular}
